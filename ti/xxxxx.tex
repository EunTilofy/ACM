% 1-8-set.tex

%%%%%%%%%%%%%%%%%%%%
\documentclass[a4paper, justified]{tufte-handout}

\input{hw-preamble} % feel free to modify this file
%%%%%%%%%%%%%%%%%%%%
\title{第8讲: 集合及其运算}
\me{徐沐杰}{211240088}{}{}
\date{\zhtoday} % or like 2019年9月13日
%%%%%%%%%%%%%%%%%%%%
\begin{document}
\maketitle
%%%%%%%%%%%%%%%%%%%%
\noplagiarism % always keep this line
%%%%%%%%%%%%%%%%%%%%
\begin{abstract}
  \mfigcap{width = 1.00\textwidth}{figs/frege}{``左边说得在理, 我深有体会''}
  \begin{center}{\fcolorbox{blue}{yellow!60}{\parbox{0.32\textwidth}{\large 
    \begin{itemize}
      \item 集合作为数学的基础
      \item 基础不牢, 地动山摇
    \end{itemize}}}}
  \end{center}
\end{abstract}
%%%%%%%%%%%%%%%%%%%%
\beginrequired

%%%%%%%%%%%%%%%
\begin{problem}[UD Problem 6.6 (f, g)]
\end{problem}

\begin{solution}
  (f) : $ (C \cap B) \setminus A$\\
  (g) : $ [(A \cap B) \cup (B \cap C) \cup (C \cap A)] \setminus (A \cap B \cap C) $\\
\end{solution}
%%%%%%%%%%%%%%%

%%%%%%%%%%%%%%%
\begin{problem}[UD Problem 7.1 (d, f)]
\end{problem}

\begin{solution}
  (d):
  \begin{itemize}
    \item If $A \subseteq B$, that means $\forall x \in X, x \in A \rightarrow x \in B$, therefore we have the contrapositive $\forall x \in X, x \notin B \rightarrow x \notin A$, so if $x \in X\setminus B$, which means $x\notin B \wedge x\in X$ , we have $x \notin A \wedge x \in X$, and thus we know that $x\in X\setminus A$, so  $X\setminus B \subseteq X\setminus A$.
    \item If $X\setminus B \subseteq X\setminus A$, that means  $\forall x \in X, x \in X\setminus B \rightarrow x\in X\setminus A$, therefore we have the contrapositive $\forall x \in X, x \notin X\setminus A \rightarrow x\notin X\setminus B$, where $x \notin X\setminus A \Leftrightarrow x \in A$ because $X$ is the universe. Similarily, we have $x \notin X\setminus B \Leftrightarrow x \in B$, and thus we have proven that $\forall x \in X, x \in A \rightarrow x \in B$, namely $A \subseteq B$.
  \end{itemize}
  (f):
  \begin{itemize}
    \item if $A\cap B = B$, then $B\subseteq A\cap B$ , $\forall x, x \in B \rightarrow x \in A \cap B \rightarrow x \in A \wedge x \in B \rightarrow x \in A$, namely $B \subseteq A$.
    \item if $B \subseteq A$, $\forall x, x \in B \rightarrow x \in A$, and thus $\forall x, x \in B \rightarrow x \in A \wedge x\in B \rightarrow x\in A \cap B$, which means $B\subseteq A\cap B$, conversely, $\forall x, x \in A \cap B \rightarrow x \in A \vee x \in B \rightarrow x \in B$, and thus $A \cap B \subseteq B$ (which is actually obvious), namely $A\cap B = B$.
  \end{itemize}
\end{solution}
%%%%%%%%%%%%%%%

%%%%%%%%%%%%%%%
\begin{problem}[UD Problem 7.2]
\end{problem}

\begin{proof}
  $$ A \cap B = \emptyset \rightarrow \forall x \in X, x \notin B \vee x \notin A $$ 
  Therefore,
  $$ \forall x \in X, x \in B \rightarrow x \notin A \rightarrow x \in (X \setminus A) $$
  Namely,
  $$ B \subseteq (X \setminus A) $$
  Reversely,
  $$ B \subseteq (X \setminus A) \rightarrow (\forall x \in X, x \in B \rightarrow x \notin A) \rightarrow (\forall x \in X, x \notin B \vee x \notin A) \rightarrow (A \cap B = \emptyset) $$
\end{proof}
%%%%%%%%%%%%%%%

%%%%%%%%%%%%%%%
\begin{problem}[UD Problem 7.14]
\end{problem}

\begin{solution}
  $$\forall x, x \in A \setminus B \rightarrow x \in A \wedge x \notin B$$
  Obviously,
  $$\forall x, x \in A \setminus B \rightarrow x \notin B $$
  Therefore,
  $$\forall x, x \notin A \setminus B \vee x \notin B $$
  Namely,
  $$\urcorner(\exists x, x \in A \setminus B \wedge x \in B) $$
  So $A \setminus B$ and $B$ are disjoint.\\
  Then consider $A \cup B$,
  $$ \forall x, x \in A \cup B \rightarrow x \in A \vee x \in B$$
  We can also say,
  $$ \forall x, x \in A \cup B \rightarrow ((x \in A \wedge x \notin B) \vee (x \in A \wedge x \in B)) \vee x \in B$$
  Therefore,
  $$ \forall x, x \in A \cup B \rightarrow (x \in A \wedge x \notin B) \vee ((x \in A \wedge x \in B) \vee x \in B)$$
  Namely,
  $$ \forall x, x \in A \cup B \rightarrow (x \in A \wedge x \notin B) \vee x \in B \rightarrow x \in (A \setminus B) \cup B $$
  Namely,
  $$ A \cup B \subseteq (A \setminus B) \cup B $$
  Reversely,
  $$\forall x, x \in (A \setminus B) \cup B \rightarrow (x \in A \wedge x \notin B) \vee x \in B $$
  Immediately we have that,
  $$\forall x, x \in (A \setminus B) \cup B \rightarrow x \in A \vee x \in B $$
  So we can say,
  $$ (A \setminus B) \cup B \subseteq A \cup B $$
  Finally,
  $$ (A \setminus B) \cup B = A \cup B $$
\end{solution}
%%%%%%%%%%%%%%%

%%%%%%%%%%%%%%%
\begin{problem}[UD Problem 7.19]
\end{problem}

\begin{proof}
  According to Problem 7.2, we know that,
  $$ A \cap (B^c \cap C^c) = \emptyset \leftrightarrow A \subseteq (B^c \cap C^c)^c$$
  According to Theorem 7.4 (15), we have,
  $$ (B^c \cap C^c)^c = B \cup C $$
  Immediately, we have,
  $$ A \cap (B^c \cap C^c) = \emptyset \leftrightarrow A \subseteq B \cup C$$
\end{proof}
%%%%%%%%%%%%%%%

%%%%%%%%%%%%%%%
\begin{problem}[UD Problem 7.20]
\end{problem}

\begin{proof}
  $$ \forall x, x \in (A\cup B) \setminus (C\cup D) \rightarrow (x \in A \vee x \in B) \wedge (x \notin C \vee x \notin D)$$
  $$ \forall x, (x \in A \wedge (x \notin C \vee x \notin D)) \vee (x \in B \wedge (x \notin C \vee x \notin D)) $$
  $$ \forall x, x \in (A \setminus (C \cup D)) \cup (B\setminus (C \cup D)) $$
  Reversely,
  $$ \forall x, x \in (A \setminus (C \cup D)) \cup (B\setminus (C \cup D))$$
  $$\forall x, (x \in A \wedge (x \notin C \vee x \notin D)) \vee (x \in B \wedge (x \notin C \vee x \notin D))$$
  $$ \forall x, (x \in A \vee x \in B) \wedge (x \notin C \vee x \notin D)$$
  $$ \forall x,  x \in (A\cup B) \setminus (C\cup D)  $$
  So,
  $$ (A\cup B) \setminus (C\cup D) =  (A \setminus (C \cup D)) \cup (B\setminus (C \cup D))$$
\end{proof}
%%%%%%%%%%%%%%%

%%%%%%%%%%%%%%%
\begin{problem}[UD Problem 8.1 (a, b)]
\end{problem}

\begin{solution}
  (a) :\\
    $[0,1)$ , $[0,1]$ , $(0,1)$ \\
  (b) :\\
    $\{0\}$ , $\{0\}$ , $\emptyset$ \\

\end{solution}
%%%%%%%%%%%%%%%

%%%%%%%%%%%%%%%
\begin{problem}[UD Problem 8.14]
\end{problem}

\begin{solution}
  It's $\mathbb{Z}$.\\
  According to Exercise 8.9, we have,
  $$ A = \mathbb{R} \setminus ( \mathbb{R} \setminus (\bigcup_{n\in \mathbb{Z}}{-n,-n+1,...,0,...,n-1,n}) ) $$
  Thus,
  $$ A = \mathbb{R} \setminus ( \mathbb{R} \setminus \mathbb{Z}) $$
  Namely,
  $$ A = \mathbb{Z} $$
\end{solution}
%%%%%%%%%%%%%%%

%%%%%%%%%%%%%%%
\begin{problem}[UD Problem 8.15]
\end{problem}

\begin{solution}
  Similarily to Problem 8.14, we have,
  $$ A = \mathbb{Q} \setminus (\mathbb{R} \setminus \bigcup_{n\in\mathbb{Z}} \{2n\} ) $$
  So we get,
  $$A= \bigcup_{n\in\mathbb{Z}} \{2n\} = \{x\in\mathbb{Z}:x=2n,n\in\mathbb{Z}\}$$
  (all even integers)
\end{solution}
%%%%%%%%%%%%%%%

%%%%%%%%%%%%%%%
\begin{problem}[UD Problem 9.8]
\end{problem}
  
\begin{proof}
  $$ A \subseteq B \rightarrow ( \forall x \in A, x \in A \rightarrow x \in B )$$
  Consider that,
  $$ \forall \alpha \in p(A), \alpha \subseteq A \rightarrow \alpha \subseteq B \rightarrow \alpha \in p(B)$$
  Thus,
  $$ p(A) \subseteq p(B) $$
  Reversely, we suppose that,
  $$ \exists x \in A, x \notin B $$
  Then,
  $$ {x} \in p(A) \wedge {x} \notin p(B)$$
  Thus,
  $$ \exists \alpha = {x} \in A, \alpha \notin A $$
  That means,
  $$ p(A) \subsetneq p(B) $$
  So,
  $$ \forall x \in A, x \in B $$
  Therefore,
  $$ p(A) \subseteq p(B) \leftrightarrow A \subseteq B $$
\end{proof}
%%%%%%%%%%%%%%%

%%%%%%%%%%%%%%%
\begin{problem}[UD Problem 9.9]
\end{problem}

\begin{proof}
  $$ \forall \beta \in \bigcup_{\alpha \in I} p(A_{\alpha}), \exists \alpha \in I, \beta \subseteq  A_{\alpha} $$
  Obviously,
  $$ \forall \alpha \in I, A_{\alpha} \subseteq \bigcup_{\alpha \in I} A_{\alpha} $$
  Hence,
  $$ \forall \beta \in \bigcup_{\alpha \in I} p(A_{\alpha}), \beta \subseteq (\bigcup_{\alpha \in I} A_{\alpha} ) \rightarrow \beta \in p(\bigcup_{\alpha \in I} A_{\alpha} ) $$
  Namely,
  $$ \bigcup_{\alpha \in I} p(A_{\alpha}) \subseteq p(\bigcup_{\alpha \in I} A_{\alpha} ) $$
\end{proof}
%%%%%%%%%%%%%%%

%%%%%%%%%%%%%%%
\begin{problem}[UD Problem 9.10]
\end{problem}

\begin{proof}
  According to Exercise 9.4, we have that,
  $$ p(A \cap B) = p(A) \cap p(B) $$
  Apply the theorem repeatedly, we have,
  $$ p(\bigcap \limits_{\alpha \in I} A_{\alpha}) = \bigcap \limits_{\alpha \in I}p(A_{\alpha}) $$
\end{proof}
%%%%%%%%%%%%%%%

%%%%%%%%%%%%%%%
\begin{problem}[改编自 UD Problem 9.19]
  请证明: 
  \[
    A \times (B \setminus C) = (A \times B) \setminus (A \times C)
  \]
\end{problem}

\begin{proof}
  $$ (x,y) \in A \times (B \setminus C) \rightarrow x \in A \wedge y\in (B\setminus C) $$
  Therefore,
  $$ x \in A \wedge (y \in B \wedge y \notin C) $$ 
  Thus,
  $$ (x \in A \wedge y \in B) \wedge (x\in A \wedge y \notin C) $$
  Namely,
  $$ (x,y) \in A \times B \wedge (x,y) \notin A \times C $$
  Namely,
  $$ (x,y) \in (A \times B) \setminus (A \times C) $$
  So we have proven that,
  $$ A \times (B \setminus C) \subseteq (A \times B) \setminus (A \times C) $$
  Reversely,
  $$ (x,y) \in (A \times B) \setminus (A \times C) \rightarrow (x,y) \in A \times B \wedge (x,y) \notin A \times C$$
  Therefore,
  $$(x \in A \wedge y \in B) \wedge (x\in A \wedge y \notin C) $$
  Immediately we have,
  $$ x \in A \wedge (y \in B \wedge y \notin C) $$
  So we get,
  $$ x \in A \times (B \setminus C) $$
  Till now we have proven that,
  $$  (A \times B) \setminus (A \times C)  \subseteq A \times (B \setminus C) $$
  Finally,
  $$ (A \times B) \setminus (A \times C) = A \times (B \setminus C) $$
\end{proof}
%%%%%%%%%%%%%%%

%%%%%%%%%%%%%%%%%%%%
\beginoptional

%%%%%%%%%%%%%%%
\begin{problem}[UD Problem 9.23]
\end{problem}

\begin{solution}
  (a):\\
  It means $ \{\{a\},\{a,b\}\} = \{\{x\},\{x,y\}\} $, we have $a=x$ and $\{a,b\}=\{x,y\}$ (namely $\{a,b\}=\{a,y\}$), therefore $a=x \wedge b=y$. \\
  (b):\\
  Since $ (a,b) = \{\{a\},\{a,b\}\} $, and $\{a\}, \{a,b\} \subseteq A\cup B $, so $\{a\}, \{a,b\} \in p(A\cup B) $, which means $\{\{a\}, \{a,b\}\} \subseteq p(p(A\cup B)) $, namely $(a,b) \subseteq p(p(A\cup B)) $.\\
  (c):\\
  There's no time left for me to prove that.
\end{solution}
%%%%%%%%%%%%%%%

%%%%%%%%%%%%%%%%%%%%
\beginot

%%%%%%%%%%%%%%%
%\begin{ot}[自然数]
%  介绍如何使用集合定义 (不限于):
%  \begin{itemize}
%    \item 自然数
%    \item 自然数上的大小关系
%    \item 自然数上的运算
%  \end{itemize}
%
%  \noindent 参考资料:
%  \begin{itemize}
%    \item \href{https://en.wikipedia.org/wiki/Natural\_number}{Natural number @ wiki}
%  \end{itemize}
%\end{ot}
%
%% \begin{solution}
%% \end{solution}
%%%%%%%%%%%%%%%%
%\vspace{0.50cm}
%%%%%%%%%%%%%%%%
%\begin{ot}[选择公理]
%  介绍选择公理 (Axiom of Choice), 如 (不限于):
%  \begin{itemize}
%    \item 不同定义形式
%    \item 怎么理解 (怎么也不理解)
%    \item 有什么用
%  \end{itemize}
%
%  \noindent 参考资料:
%  \begin{itemize}
%    \item \href{https://en.wikipedia.org/wiki/Axiom\_of\_choice}{Axiom of choice @ wiki}
%    \item \href{https://plato.stanford.edu/entries/axiom-choice/}{The Axiom of Choice @ Stanford Encyclopedia of Philosophy}
%  \end{itemize}
%\end{ot}

% \begin{solution}
% \end{solution}
%%%%%%%%%%%%%%%

%%%%%%%%%%%%%%%%%%%%
% 如果没有需要订正的题目,可以把这部分删掉

\begincorrection

%%%%%%%%%%%%%%%%%%%%

%%%%%%%%%%%%%%%%%%%%
% 如果没有反馈,可以把这部分删掉
\beginfb

% 你可以写
% ~\footnote{优先推荐 \href{problemoverflow.top}{ProblemOverflow}}:
% \begin{itemize}
%   \item 对课程及教师的建议与意见
%   \item 教材中不理解的内容
%   \item 希望深入了解的内容
%   \item $\cdots$
% \end{itemize}
%%%%%%%%%%%%%%%%%%%%
\end{document}